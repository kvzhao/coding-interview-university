%
% 6.006 problem set 7
%
\documentclass[12pt,twoside]{article}

% Cross-references for handout numbers.

% Updated to include SMA course for Fall 2001 -- cel

\newcommand{\name}{}


\usepackage{latexsym}
%\usepackage{bbm}
\usepackage{times,url}
\usepackage{clrscode3e}

\newcommand{\mitst}[1]{\begin{description}
\item[MIT students:] #1
\end{description}}
\newcommand{\smast}[1]{\begin{description}
\item[SMA students:] #1
\end{description}}

\newcommand{\profs}{Professors Erik Demaine and Srini Devadas}
\newcommand{\subj}{6.006}

\newlength{\toppush}
\setlength{\toppush}{2\headheight}
\addtolength{\toppush}{\headsep}

\newcommand{\htitle}[2]{\noindent\vspace*{-\toppush}\newline\parbox{6.5in}
{\textit{Introduction to Algorithms: 6.006}\hfill\name\newline
Massachusetts Institute of Technology \hfill #2\newline
\profs\hfill #1 \vspace*{-.5ex}\newline
\mbox{}\hrulefill\mbox{}}\vspace*{1ex}\mbox{}\newline
\begin{center}{\Large\bf #1}\end{center}}

\newcommand{\handout}[2]{\thispagestyle{empty}
 \markboth{#1}{#1}
 \pagestyle{myheadings}\htitle{#1}{#2}}

\newcommand{\htitlewithouttitle}[2]{\noindent\vspace*{-\toppush}\newline\parbox{6.5in}
{\textit{Introduction to Algorithms}\hfill#2\newline
Massachusetts Institute of Technology \hfill 6.006\newline
%Singapore-MIT Alliance \hfill SMA5503\newline
\profs\hfill Handout #1\vspace*{-.5ex}\newline
\mbox{}\hrulefill\mbox{}}\vspace*{1ex}\mbox{}\newline}

\newcommand{\handoutwithouttitle}[2]{\thispagestyle{empty}
 \markboth{Handout \protect\ref{#1}}{Handout \protect\ref{#1}}
 \pagestyle{myheadings}\htitlewithouttitle{\protect\ref{#1}}{#2}}

\newcommand{\exam}[2]{% parameters: exam name, date
 \thispagestyle{empty}
 \markboth{\subj\ #1\hspace{1in}Name\hrulefill\ \ }%
          {\subj\ #1\hspace{1in}Name\hrulefill\ \ }
 \pagestyle{myheadings}\examtitle{#1}{#2}
 \renewcommand{\theproblem}{Problem \arabic{problemnum}}
}
\newcommand{\examsolutions}[3]{% parameters: handout, exam name, date
 \thispagestyle{empty}
 \markboth{Handout \protect\ref{#1}: #2}{Handout \protect\ref{#1}: #2}
% \pagestyle{myheadings}\htitle{\protect\ref{#1}}{#2}{#3}
 \pagestyle{myheadings}\examsolutionstitle{\protect\ref{#1}} {#2}{#3}
 \renewcommand{\theproblem}{Problem \arabic{problemnum}}
}
\newcommand{\examsolutionstitle}[3]{\noindent\vspace*{-\toppush}\newline\parbox{6.5in}
{\textit{Introduction to Algorithms}\hfill#3\newline
Massachusetts Institute of Technology \hfill 6.006\newline
%Singapore-MIT Alliance \hfill SMA5503\newline
\profs\hfill Handout #1\vspace*{-.5ex}\newline
\mbox{}\hrulefill\mbox{}}\vspace*{1ex}\mbox{}\newline
\begin{center}{\Large\bf #2}\end{center}}

\newcommand{\takehomeexam}[2]{% parameters: exam name, date
 \thispagestyle{empty}
 \markboth{\subj\ #1\hfill}{\subj\ #1\hfill}
 \pagestyle{myheadings}\examtitle{#1}{#2}
 \renewcommand{\theproblem}{Problem \arabic{problemnum}}
}

\makeatletter
\newcommand{\exambooklet}[2]{% parameters: exam name, date
 \thispagestyle{empty}
 \markboth{\subj\ #1}{\subj\ #1}
 \pagestyle{myheadings}\examtitle{#1}{#2}
 \renewcommand{\theproblem}{Problem \arabic{problemnum}}
 \renewcommand{\problem}{\newpage
 \item \let\@currentlabel=\theproblem
 \markboth{\subj\ #1, \theproblem}{\subj\ #1, \theproblem}}
}
\makeatother


\newcommand{\examtitle}[2]{\noindent\vspace*{-\toppush}\newline\parbox{6.5in}
{\textit{Introduction to Algorithms}\hfill#2\newline
Massachusetts Institute of Technology \hfill 6.006 Fall 2009\newline
%Singapore-MIT Alliance \hfill SMA5503\newline
\profs\hfill #1\vspace*{-.5ex}\newline
\mbox{}\hrulefill\mbox{}}\vspace*{1ex}\mbox{}\newline
\begin{center}{\Large\bf #1}\end{center}}

\newcommand{\grader}[1]{\hspace{1cm}\textsf{\textbf{#1}}\hspace{1cm}}

%\newcommand{\points}[1]{[#1 points]\ }
\newcount\pointcount
\newcommand{\points}[1]{\pointcount=#1\relax \ifnum\pointcount=1 [#1 point]\else [#1 points]\fi\ }

\newcommand{\parts}[1]
{
  \ifnum#1=1
  (1 part)
  \else
  (#1 parts)
  \fi
  \ 
}

\newcommand{\bparts}{\begin{problemparts}}
\newcommand{\eparts}{\end{problemparts}}
\newcommand{\ppart}{\problempart}

%\newcommand{\lg} {lg\ }

\setlength{\oddsidemargin}{0pt}
\setlength{\evensidemargin}{0pt}
\setlength{\textwidth}{6.5in}
\setlength{\topmargin}{0in}
\setlength{\textheight}{8.5in}


\renewcommand{\cases}[1]{\left\{ \begin{array}{ll}#1\end{array}\right.}
\newcommand{\cif}[1]{\mbox{if $#1$}}
\newcommand{\cwhen}[1]{\mbox{when $#1$}}

\newcounter{problemnum}
\newcommand{\theproblem}{Problem \theproblemsetnum-\arabic{problemnum}}
\newenvironment{problems}{
        \begin{list}{{\bf \theproblem. \hspace*{0.5em}}}
        {\setlength{\leftmargin}{0em}
         \setlength{\rightmargin}{0em}
         \setlength{\labelwidth}{0em}
         \setlength{\labelsep}{0em}
         \usecounter{problemnum}}}{\end{list}}
\makeatletter
\newcommand{\problem}[1][{}]{\item \let\@currentlabel=\theproblem \textbf{#1}}
\makeatother

\newcounter{problempartnum}[problemnum]
\newenvironment{problemparts}{
        \begin{list}{{\bf (\alph{problempartnum})}}
        {\setlength{\leftmargin}{2.5em}
         \setlength{\rightmargin}{2.5em}
         \setlength{\labelsep}{0.5em}}}{\end{list}}
\newcommand{\problempart}{\addtocounter{problempartnum}{1}\item}

\newenvironment{truefalseproblemparts}{
        \begin{list}{{\bf (\alph{problempartnum})\ \ \ T\ \ F\hfil}}
        {\setlength{\leftmargin}{4.5em}
         \setlength{\rightmargin}{2.5em}
         \setlength{\labelsep}{0.5em}
         \setlength{\labelwidth}{4.5em}}}{\end{list}}

\newcounter{exercisenum}
\newcommand{\theexercise}{Exercise \theproblemsetnum-\arabic{exercisenum}}
\newenvironment{exercises}{
        \begin{list}{{\bf \theexercise. \hspace*{0.5em}}}
        {\setlength{\leftmargin}{0em}
         \setlength{\rightmargin}{0em}
         \setlength{\labelwidth}{0em}
         \setlength{\labelsep}{0em}
        \usecounter{exercisenum}}}{\end{list}}
\makeatletter
\newcommand{\exercise}{\item \let\@currentlabel=\theexercise}
\makeatother

\newcounter{exercisepartnum}[exercisenum]
%\newcommand{\problem}[1]{\medskip\mbox{}\newline\noindent{\bf Problem #1.}\hspace*{1em}}
%\newcommand{\exercise}[1]{\medskip\mbox{}\newline\noindent{\bf Exercise #1.}\hspace*{1em}}

\newenvironment{exerciseparts}{
        \begin{list}{{\bf (\alph{exercisepartnum})}}
        {\setlength{\leftmargin}{2.5em}
         \setlength{\rightmargin}{2.5em}
         \setlength{\labelsep}{0.5em}}}{\end{list}}
\newcommand{\exercisepart}{\addtocounter{exercisepartnum}{1}\item}


% Macros to make captions print with small type and 'Figure xx' in bold.
\makeatletter
\def\fnum@figure{{\bf Figure \thefigure}}
\def\fnum@table{{\bf Table \thetable}}
\let\@mycaption\caption
%\long\def\@mycaption#1[#2]#3{\addcontentsline{\csname
%  ext@#1\endcsname}{#1}{\protect\numberline{\csname 
%  the#1\endcsname}{\ignorespaces #2}}\par
%  \begingroup
%    \@parboxrestore
%    \small
%    \@makecaption{\csname fnum@#1\endcsname}{\ignorespaces #3}\par
%  \endgroup}
%\def\mycaption{\refstepcounter\@captype \@dblarg{\@mycaption\@captype}}
%\makeatother
\let\mycaption\caption
%\newcommand{\figcaption}[1]{\mycaption[]{#1}}

\newcounter{totalcaptions}
\newcounter{totalart}

\newcommand{\figcaption}[1]{\addtocounter{totalcaptions}{1}\caption[]{#1}}

% \psfigures determines what to do for figures:
%       0 means just leave vertical space
%       1 means put a vertical rule and the figure name
%       2 means insert the PostScript version of the figure
%       3 means put the figure name flush left or right
\newcommand{\psfigures}{0}
\newcommand{\spacefigures}{\renewcommand{\psfigures}{0}}
\newcommand{\rulefigures}{\renewcommand{\psfigures}{1}}
\newcommand{\macfigures}{\renewcommand{\psfigures}{2}}
\newcommand{\namefigures}{\renewcommand{\psfigures}{3}}

\newcommand{\figpart}[1]{{\bf (#1)}\nolinebreak[2]\relax}
\newcommand{\figparts}[2]{{\bf (#1)--(#2)}\nolinebreak[2]\relax}


\macfigures     % STATE

% When calling \figspace, make sure to leave a blank line afterward!!
% \widefigspace is for figures that are more than 28pc wide.
\newlength{\halffigspace} \newlength{\wholefigspace}
\newlength{\figruleheight} \newlength{\figgap}
\newcommand{\setfiglengths}{\ifnum\psfigures=1\setlength{\figruleheight}{\hruleheight}\setlength{\figgap}{1em}\else\setlength{\figruleheight}{0pt}\setlength{\figgap}{0em}\fi}
\newcommand{\figspace}[2]{\ifnum\psfigures=0\leavefigspace{#1}\else%
\setfiglengths%
\setlength{\wholefigspace}{#1}\setlength{\halffigspace}{.5\wholefigspace}%
\rule[-\halffigspace]{\figruleheight}{\wholefigspace}\hspace{\figgap}#2\fi}
\newlength{\widefigspacewidth}
% Make \widefigspace put the figure flush right on the text page.
\newcommand{\widefigspace}[2]{
\ifnum\psfigures=0\leavefigspace{#1}\else%
\setfiglengths%
\setlength{\widefigspacewidth}{28pc}%
\addtolength{\widefigspacewidth}{-\figruleheight}%
\setlength{\wholefigspace}{#1}\setlength{\halffigspace}{.5\wholefigspace}%
\makebox[\widefigspacewidth][r]{#2\hspace{\figgap}}\rule[-\halffigspace]{\figruleheight}{\wholefigspace}\fi}
\newcommand{\leavefigspace}[1]{\setlength{\wholefigspace}{#1}\setlength{\halffigspace}{.5\wholefigspace}\rule[-\halffigspace]{0em}{\wholefigspace}}

% Commands for including figures with macpsfig.
% To use these commands, documentstyle ``macpsfig'' must be specified.
\newlength{\macfigfill}
\makeatother
\newlength{\bbx}
\newlength{\bby}
\newcommand{\macfigure}[5]{\addtocounter{totalart}{1}
\ifnum\psfigures=2%
\setlength{\bbx}{#2}\addtolength{\bbx}{#4}%
\setlength{\bby}{#3}\addtolength{\bby}{#5}%
\begin{flushleft}
\ifdim#4>28pc\setlength{\macfigfill}{#4}\addtolength{\macfigfill}{-28pc}\hspace*{-\macfigfill}\fi%
\mbox{\psfig{figure=./#1.ps,%
bbllx=#2,bblly=#3,bburx=\bbx,bbury=\bby}}
\end{flushleft}%
\else\ifdim#4>28pc\widefigspace{#5}{#1}\else\figspace{#5}{#1}\fi\fi}
\makeatletter

\newlength{\savearraycolsep}
\newcommand{\narrowarray}[1]{\setlength{\savearraycolsep}{\arraycolsep}\setlength{\arraycolsep}{#1\arraycolsep}}
\newcommand{\normalarray}{\setlength{\arraycolsep}{\savearraycolsep}}

\newcommand{\hint}{{\em Hint:\ }}

% Macros from /th/u/clr/mac.tex

\newcommand{\set}[1]{\left\{ #1 \right\}}
\newcommand{\abs}[1]{\left| #1\right|}
\newcommand{\card}[1]{\left| #1\right|}
\newcommand{\floor}[1]{\left\lfloor #1 \right\rfloor}
\newcommand{\ceil}[1]{\left\lceil #1 \right\rceil}
\newcommand{\ang}[1]{\ifmmode{\left\langle #1 \right\rangle}
   \else{$\left\langle${#1}$\right\rangle$}\fi}
        % the \if allows use outside mathmode,
        % but will swallow following space there!
\newcommand{\paren}[1]{\left( #1 \right)}
\newcommand{\bracket}[1]{\left[ #1 \right]}
\newcommand{\prob}[1]{\Pr\left\{ #1 \right\}}
\newcommand{\Var}{\mathop{\rm Var}\nolimits}
\newcommand{\expect}[1]{{\rm E}\left[ #1 \right]}
\newcommand{\expectsq}[1]{{\rm E}^2\left[ #1 \right]}
\newcommand{\variance}[1]{{\rm Var}\left[ #1 \right]}
\renewcommand{\choose}[2]{{{#1}\atopwithdelims(){#2}}}
\def\pmod#1{\allowbreak\mkern12mu({\rm mod}\,\,#1)}
\newcommand{\matx}[2]{\left(\begin{array}{*{#1}{c}}#2\end{array}\right)}
\newcommand{\Adj}{\mathop{\rm Adj}\nolimits}

\newtheorem{theorem}{Theorem}
\newtheorem{lemma}[theorem]{Lemma}
\newtheorem{corollary}[theorem]{Corollary}
\newtheorem{xample}{Example}
\newtheorem{definition}{Definition}
\newenvironment{example}{\begin{xample}\rm}{\end{xample}}
\newcommand{\proof}{\noindent{\em Proof.}\hspace{1em}}
\def\squarebox#1{\hbox to #1{\hfill\vbox to #1{\vfill}}}
\newcommand{\qedbox}{\vbox{\hrule\hbox{\vrule\squarebox{.667em}\vrule}\hrule}}
\newcommand{\qed}{\nopagebreak\mbox{}\hfill\qedbox\smallskip}
\newcommand{\eqnref}[1]{(\protect\ref{#1})}

%%\newcommand{\twodots}{\mathinner{\ldotp\ldotp}}
\newcommand{\transpose}{^{\mbox{\scriptsize \sf T}}}
\newcommand{\amortized}[1]{\widehat{#1}}

\newcommand{\punt}[1]{}

%%% command for putting definitions into boldface
% New style for defined terms, as of 2/23/88, redefined by THC.
\newcommand{\defn}[1]{{\boldmath\textit{\textbf{#1}}}}
\newcommand{\defi}[1]{{\textit{\textbf{#1\/}}}}

\newcommand{\red}{\leq_{\rm P}}
\newcommand{\lang}[1]{%
\ifmmode\mathord{\mathcode`-="702D\rm#1\mathcode`\-="2200}\else{\rm#1}\fi}

%\newcommand{\ckt}[1]{\ifmmode\mathord{\mathcode`-="702D\sc #1\mathcode`\-="2200}\else$\mathord{\mathcode`-="702D\sc #1\mathcode`\-="2200}$\fi}
\newcommand{\ckt}[1]{\ifmmode \sc #1\else$\sc #1$\fi}

%% Margin notes - use \notesfalse to turn off notes.
\setlength{\marginparwidth}{0.6in}
\reversemarginpar
\newif\ifnotes
\notestrue
\newcommand{\longnote}[1]{
  \ifnotes
    {\medskip\noindent Note: \marginpar[\hfill$\Longrightarrow$]
      {$\Longleftarrow$}{#1}\medskip}
  \fi}
\newcommand{\note}[1]{
  \ifnotes
    {\marginpar{\tiny \raggedright{#1}}}
  \fi}


\newcommand{\reals}{\mathbbm{R}}
\newcommand{\integers}{\mathbbm{Z}}
\newcommand{\naturals}{\mathbbm{N}}
\newcommand{\rationals}{\mathbbm{Q}}
\newcommand{\complex}{\mathbbm{C}}

\newcommand{\oldreals}{{\bf R}}
\newcommand{\oldintegers}{{\bf Z}}
\newcommand{\oldnaturals}{{\bf N}}
\newcommand{\oldrationals}{{\bf Q}}
\newcommand{\oldcomplex}{{\bf C}}

\newcommand{\w}{\omega}                 %% for fft chapter

\newenvironment{closeitemize}{\begin{list}
{$\bullet$}
{\setlength{\itemsep}{-0.2\baselineskip}
\setlength{\topsep}{0.2\baselineskip}
\setlength{\parskip}{0pt}}}
{\end{list}}

% These are necessary within a {problems} environment in order to restore
% the default separation between bullets and items.
\newenvironment{normalitemize}{\setlength{\labelsep}{0.5em}\begin{itemize}}
                              {\end{itemize}}
\newenvironment{normalenumerate}{\setlength{\labelsep}{0.5em}\begin{enumerate}}
                                {\end{enumerate}}

%\def\eqref#1{Equation~(\ref{eq:#1})}
%\newcommand{\eqref}[1]{Equation (\ref{eq:#1})}
\newcommand{\eqreftwo}[2]{Equations (\ref{eq:#1}) and~(\ref{eq:#2})}
\newcommand{\ineqref}[1]{Inequality~(\ref{ineq:#1})}
\newcommand{\ineqreftwo}[2]{Inequalities (\ref{ineq:#1}) and~(\ref{ineq:#2})}

\newcommand{\figref}[1]{Figure~\ref{fig:#1}}
\newcommand{\figreftwo}[2]{Figures \ref{fig:#1} and~\ref{fig:#2}}

\newcommand{\liref}[1]{line~\ref{li:#1}}
\newcommand{\Liref}[1]{Line~\ref{li:#1}}
\newcommand{\lirefs}[2]{lines \ref{li:#1}--\ref{li:#2}}
\newcommand{\Lirefs}[2]{Lines \ref{li:#1}--\ref{li:#2}}
\newcommand{\lireftwo}[2]{lines \ref{li:#1} and~\ref{li:#2}}
\newcommand{\lirefthree}[3]{lines \ref{li:#1}, \ref{li:#2}, and~\ref{li:#3}}

\newcommand{\lemlabel}[1]{\label{lem:#1}}
\newcommand{\lemref}[1]{Lemma~\ref{lem:#1}} 

\newcommand{\exref}[1]{Exercise~\ref{ex:#1}}

\newcommand{\handref}[1]{Handout~\ref{#1}}

\newcommand{\defref}[1]{Definition~\ref{def:#1}}

% (1997.8.16: Victor Luchangco)
% Modified \hlabel to only get date and to use handouts counter for number.
%   New \handout and \handoutwithouttitle commands in newmac.tex use this.
%   The date is referenced by <label>-date.
%   (Retained old definition as \hlabelold.)
%   Defined \hforcelabel to use an argument instead of the handouts counter.

\newcounter{handouts}
\setcounter{handouts}{0}

\newcommand{\hlabel}[2]{%
\stepcounter{handouts}
{\edef\next{\write\@auxout{\string\newlabel{#1}{{\arabic{handouts}}{0}}}}\next}
\write\@auxout{\string\newlabel{#1-date}{{#2}{0}}}
}

\newcommand{\hforcelabel}[3]{%          Does not step handouts counter.
\write\@auxout{\string\newlabel{#1}{{#2}{0}}}
\write\@auxout{\string\newlabel{#1-date}{{#3}{0}}}}


% less ugly underscore
% --juang, 2008 oct 05
\renewcommand{\_}{\vrule height 0 pt depth 0.4 pt width 0.5 em \,}


\usepackage{amsmath}
\usepackage{url}
\usepackage{mdwlist}
\usepackage{graphicx}
\usepackage{clrscode3e}
\newcommand{\isnotequal}{\mathrel{\scalebox{0.8}[1]{!}\hspace*{1pt}\scalebox{0.8}[1]{=}}}
\usepackage{listings}
\usepackage{tikz}
\usepackage{CJKutf8}

\setlength{\oddsidemargin}{0pt}
\setlength{\evensidemargin}{0pt}
\setlength{\textwidth}{6.5in}
\setlength{\topmargin}{0in}
\setlength{\textheight}{8.5in}

\newcommand{\theproblemsetnum}{7}
\newcommand{\releasedate}{November 22, 2011}
\newcommand{\partaduedate}{Tuesday, December 6}
\newcommand{\tabUnit}{3ex}
\newcommand{\tabT}{\hspace*{\tabUnit}}

\newcommand{\collaborators}{
%%% COLLABORATORS START %%%
None.
%%% COLLABORATORS END %%%
}

\newcommand{\answerIIaDale}{
%%% PROBLEM 2(a) DALE ANSWER START %%%
0
%%% PROBLEM 2(a) DALE ANSWER END %%%
}

\newcommand{\answerIIaJCN}{
%%% PROBLEM 2(a) JCN ANSWER START %%%
0
%%% PROBLEM 2(a) JCN ANSWER END %%%
}

\newcommand{\answerIIaMacroware}{
%%% PROBLEM 2(a) MACROWARE ANSWER START %%%
0
%%% PROBLEM 2(a) MACROWARE ANSWER END %%%
}

\newcommand{\answerIIaPear}{
%%% PROBLEM 2(a) PEAR ANSWER START %%%
0
%%% PROBLEM 2(a) PEAR ANSWER END %%%
}

\newcommand{\answerIIbDale}{
%%% PROBLEM 2(b) DALE ANSWER START %%%
0
%%% PROBLEM 2(b) DALE ANSWER END %%%
}

\newcommand{\answerIIbJCN}{
%%% PROBLEM 2(b) JCN ANSWER START %%%
0
%%% PROBLEM 2(b) JCN ANSWER END %%%
}

\newcommand{\answerIIbMacroware}{
%%% PROBLEM 2(b) MACROWARE ANSWER START %%%
0
%%% PROBLEM 2(b) MACROWARE ANSWER END %%%
}

\newcommand{\answerIIbPear}{
%%% PROBLEM 2(b) PEAR ANSWER START %%%
0
%%% PROBLEM 2(b) PEAR ANSWER END %%%
}

\newcommand{\answerIIcDale}{
%%% PROBLEM 2(c) DALE ANSWER START %%%
0
%%% PROBLEM 2(c) DALE ANSWER END %%%
}

\newcommand{\answerIIcJCN}{
%%% PROBLEM 2(c) JCN ANSWER START %%%
0
%%% PROBLEM 2(c) JCN ANSWER END %%%
}

\newcommand{\answerIIcMacroware}{
%%% PROBLEM 2(c) MACROWARE ANSWER START %%%
0
%%% PROBLEM 2(c) MACROWARE ANSWER END %%%
}

\newcommand{\answerIIcPear}{
%%% PROBLEM 2(c) PEAR ANSWER START %%%
0
%%% PROBLEM 2(c) PEAR ANSWER END %%%
}

\newcommand{\answerIId}{\solution{
%%% PROBLEM 2(d) ANSWER START %%%
0
%%% PROBLEM 2(d) ANSWER END %%%
}}

\newcommand{\answerIIe}{\solution{
%%% PROBLEM 2(e) ANSWER START %%%
0
%%% PROBLEM 2(e) ANSWER END %%%
}}

\newcommand{\answerIIf}{\solution{
%%% PROBLEM 2(f) ANSWER START %%%
0
%%% PROBLEM 2(f) ANSWER END %%%
}}

\newcommand{\answerIIg}{\solution{
%%% PROBLEM 2(g) ANSWER START %%%
0
%%% PROBLEM 2(g) ANSWER END %%%
}}

\newcommand{\answerIIh}{\solution{
%%% PROBLEM 2(h) ANSWER START %%%
0 0 0
%%% PROBLEM 2(h) ANSWER END %%%
}}

\newcommand{\answerIIi}{\solution{
%%% PROBLEM 2(i) ANSWER START %%%
0
%%% PROBLEM 2(i) ANSWER END %%%
}}

\newcommand{\answerIIj}{\solution{
%%% PROBLEM 2(j) ANSWER START %%%
0
%%% PROBLEM 2(j) ANSWER END %%%
}}

\newcommand{\answerIIk}{\solution{
%%% PROBLEM 2(k) ANSWER START %%%
0
%%% PROBLEM 2(k) ANSWER END %%%
}}

\newcommand{\answerIIl}{\solution{
%%% PROBLEM 2(l) ANSWER START %%%
0
%%% PROBLEM 2(l) ANSWER END %%%
}}

\newcommand{\answerIIm}{\solution{
%%% PROBLEM 2(m) ANSWER START %%%
0
%%% PROBLEM 2(m) ANSWER END %%%
}}

\newcommand{\answerIIn}{\solution{
%%% PROBLEM 2(n) ANSWER START %%%
0
%%% PROBLEM 2(n) ANSWER END %%%
}}

\newcommand{\answerIIo}{\solution{
%%% PROBLEM 2(o) ANSWER START %%%
0
%%% PROBLEM 2(o) ANSWER END %%%
}}

\newcommand{\answerIIp}{\solution{
%%% PROBLEM 2(p) ANSWER START %%%
0
%%% PROBLEM 2(p) ANSWER END %%%
}}

\newcommand{\answerIIq}{\solution{
%%% PROBLEM 2(q) ANSWER START %%%
0
%%% PROBLEM 2(q) ANSWER END %%%
}}

\newcommand{\answerIIrDale}{
%%% PROBLEM 2(r) DALE ANSWER START %%%
0
%%% PROBLEM 2(r) DALE ANSWER END %%%
}

\newcommand{\answerIIrJCN}{
%%% PROBLEM 2(r) JCN ANSWER START %%%
0
%%% PROBLEM 2(r) JCN ANSWER END %%%
}

\newcommand{\answerIIrMacroware}{
%%% PROBLEM 2(r) MACROWARE ANSWER START %%%
0
%%% PROBLEM 2(r) MACROWARE ANSWER END %%%
}

\newcommand{\answerIIrPear}{
%%% PROBLEM 2(r) PEAR ANSWER START %%%
0
%%% PROBLEM 2(r) PEAR ANSWER END %%%
}

\newcommand{\answerIIsDale}{
%%% PROBLEM 2(s) DALE ANSWER START %%%
0
%%% PROBLEM 2(s) DALE ANSWER END %%%
}

\newcommand{\answerIIsJCN}{
%%% PROBLEM 2(s) JCN ANSWER START %%%
0
%%% PROBLEM 2(s) JCN ANSWER END %%%
}

\newcommand{\answerIIsMacroware}{
%%% PROBLEM 2(s) MACROWARE ANSWER START %%%
0
%%% PROBLEM 2(s) MACROWARE ANSWER END %%%
}

\newcommand{\answerIIsPear}{
%%% PROBLEM 2(s) PEAR ANSWER START %%%
0
%%% PROBLEM 2(s) PEAR ANSWER END %%%
}

\newcommand{\answerIIt}{\solution{
%%% PROBLEM 2(t) ANSWER START %%%
\begin{codebox}
\Procname{$\proc{StockLimited}(\total, \stockCount, \priceStart, \priceEnd, \limit)$}
\li Insert your pseudocode here.
\end{codebox}

Remember to analyze your pseudocode as well.
%%% PROBLEM 2(t) ANSWER END %%%
}}

\begin{document}

\handout{Problem Set \theproblemsetnum}{\releasedate}

\newcommand{\solution}[1]{
  \par\medskip
  \textbf{Answer:} #1
}

\textbf{Both theory and programming questions} are due on {\bf \partaduedate} at
{\bf 11:59PM}.
%
Please download the .zip archive for this problem set.
Refer to the
\texttt{README.txt} file for instructions on preparing your solutions.

We will provide the solutions to the problem set 10 hours after the problem set
is due. You will have to read the solutions, and write a brief \textbf{grading
explanation} to help your grader understand your write-up. You will need to
submit the grading guide by \textbf{Thursday, December 8, 11:59PM}. Your grade
will be based on both your solutions and the grading explanation.

\medskip

\noindent
\hrulefill

\medskip

\textbf{Collaborators:} \collaborators

\noindent
\hrulefill

\begin{problems}


\problem \points{30} \textbf{Seam Carving}



  In a recent paper, ``Seam Carving for Content-Aware Image
  Resizing'', Shai Avidan and Ariel Shamir describe a novel method of
  resizing images. You are welcome to read the paper, but we recommend
  starting with the YouTube video:

  \centerline{\url{http://www.youtube.com/watch?v=vIFCV2spKtg}}

  Both are linked
  from the Problem Sets page on the class website. After you've
  watched the video, the terminology in the rest of this problem will
  make sense.

  If you were paying attention around time 1:50 of the video, then you
  can probably guess what you're going to have to do. You are given an
  image, and your task is to calculate the best vertical seam to
  remove. A \emph{vertical seam} is a connected path of pixels, one
  pixel in each row. We call two pixels \emph{connected} if they are
  vertically or diagonally adjacent. The \emph{best} vertical seam is
  the one that minimizes the total ``energy'' of pixels in the seam.

  The video didn't spend much time on dynamic programming,
  so here's the algorithm:

  \begin{description}
\setlength{\labelsep}{0.5em}
  \item[Subproblems:] For each pixel $(i,j)$, what is the lower-energy
    seam that starts at the top row of the image, but ends at $(i,j)$?
  \item[Relation:] Let \verb|dp[i,j]| be the solution to subproblem
  $(i,j)$. Then \\
  \verb|dp[i,j] = min(dp[i,j-1],dp[i-1,j-1],dp[i+1,j-1]) + energy(i,j)|
  \item[Analysis:] Solving each subproblem takes $O(1)$ time: there
    are three smaller subproblems to look up, and one call to
    \verb|energy()|, which all take $O(1)$ time. There is one
    subproblem for each pixel, so the running time is $\Theta(A)$,
    where $A$ is the number of pixels, i.e., the area of the image.
  \end{description}

  Download \verb|ps7_code.zip| and unpack it.
  To solve this problem set,
  you will need the Python Imaging Library (PIL),
  which you should have installed for Problem Set 4.
  If you wish to view your results,
  you will additionally need the Tkinter library.

  In \verb|resizeable_image.py|, write a function
  \verb|best_seam(self)| that returns a list of coordinates
  corresponding to the cheapest vertical seam to remove, e.g.,
  $[(5,0), (5,1), (4,2), (5,3), (6,4)]$.  You should implement the
  dynamic program described above in a bottom-up manner.

  The class \verb|ResizeableImage|
  inherits from \verb|ImageMatrix|.
  You should use the following components
  of \verb|ImageMatrix| in your dynamic
  program:
  \begin{itemize}
\setlength{\labelsep}{0.5em}
  \item \verb|self.energy(i,j)| returns the energy of a pixel. This
  takes $O(1)$ time, but the constant factor is sizeable.  If you call it
  more than once, you might want to cache the results.
  \item \verb|self.width| and \verb|self.height| are the width and height
        of the image, respectively.
  \end{itemize}

  Test your code using \verb|test_resizable_image.py|, and submit
  \verb|ResizeableImage.py| to the class website. You can also view
  your code in action by running \verb|gui.py|. Included with the
  problem set are two differently sized versions of the same sunset
  image. If you remove enough seams from the sunset image, it should
  center the sun.

  Also, please try out your own pictures (most file formats should
  work), and send us any interesting before/after shots.

\newpage

\newcommand{\total}{\id{total}}
\newcommand{\stockCount}{\id{count}}
\newcommand{\priceStart}{\id{start}}
\newcommand{\priceEnd}{\id{end}}
\newcommand{\limit}{\id{limit}}

\newcommand{\itemCount}{\id{items}}
\newcommand{\itemSize}{\id{size}}
\newcommand{\itemValue}{\id{value}}
\newcommand{\capacity}{\id{capacity}}

\newcommand{\profit}{\id{profit}}
\newcommand{\purchase}{\id{purchase}}
\newcommand{\cash}{\id{cash}}
\newcommand{\current}{\id{current}}
\newcommand{\stock}{\id{stock}}
\newcommand{\result}{\id{result}}
\newcommand{\quantity}{\id{quantity}}
\newcommand{\leftover}{\id{leftover}}

\newcommand{\priceTable}{
\begin{center}
\begin{tabular}{|l|c|c|}
\hline
\bf Company & \bf Price in 1991 & \bf Price in 2011 \\ \hline
Dale, Inc. & \$12 & \$39 \\
JCN Corp. & \$10 & \$13 \\
Macroware, Inc. & \$18 & \$47 \\
Pear, Inc. & \$15 & \$45 \\ \hline
\end{tabular}
\end{center}
}

\problem \points{70} \textbf{HG Fargo}

You have been given an internship
at the extremely profitable and secretive bank
HG Fargo.
Your immediate supervisor tells you that
higher-ups in the bank are very interested
in learning from the past.
In particular,
they want to know
how much money they \emph{could} have made
if they had invested optimally.

Your supervisor gives you the following data
on the prices%
\footnote{
Note that for the purposes of this problem,
you should ignore some of the intricacies of the real stock market.
The only income
you can make is from purchasing stocks in 1991,
then selling those same stocks at market value
in 2011.
}
of select stocks
in 1991 and in 2011:
\priceTable
As a first step,
you decide to examine what the optimal decision is
for a couple of small examples:

\begin{problemparts}

\problempart
\points{5}
If you had \$20 available to purchase stocks in 1991,
how much of each stock should you have bought
to maximize profits when you sell everything in 2011?
Note that you do not need to invest all of your money
--- if it is more profitable to keep some as cash,
you do not need to invest it.

\solution{

\begin{center}
\begin{tabular}{|c|c|}
\hline
\bf Company & \bf Number of Shares \\
\hline
Dale, Inc. & \answerIIaDale \\
JCN Corp. & \answerIIaJCN \\
Macroware, Inc. & \answerIIaMacroware \\
Pear, Inc. & \answerIIaPear \\
\hline
\end{tabular}
\end{center}
}

\problempart
\points{5}
If you had \$30 available to purchase stocks in 1991,
how much of each stock should you have bought?
\solution{

\begin{center}
\begin{tabular}{|c|c|}
\hline
\bf Company & \bf Number of Shares \\
\hline
Dale, Inc. & \answerIIbDale \\
JCN Corp. & \answerIIbJCN \\
Macroware, Inc. & \answerIIbMacroware \\
Pear, Inc. & \answerIIbPear \\
\hline
\end{tabular}
\end{center}
}

\problempart
\points{5}
If you had \$120 available to purchase stocks in 1991,
how much of each stock should you have bought?
\solution{

\begin{center}
\begin{tabular}{|c|c|}
\hline
\bf Company & \bf Number of Shares \\
\hline
Dale, Inc. & \answerIIcDale \\
JCN Corp. & \answerIIcJCN \\
Macroware, Inc. & \answerIIcMacroware \\
Pear, Inc. & \answerIIcPear \\
\hline
\end{tabular}
\end{center}
}

\end{problemparts}

Your supervisor asks you to write an algorithm
for computing the best way to purchase stocks,
given the initial money $\total$,
the number $\stockCount$ of companies with stock available,
an array $\priceStart$ containing the prices of each stock in 1991,
and an array $\priceEnd$ containing the prices of each stock in 2011.
All prices are assumed to be positive integers.

There is a strong relationship
between this problem
and the knapsack problem.
The knapsack problem takes four inputs:
the number of different items $\itemCount$,
the item sizes $\itemSize$ (all of which are integers),
the item values $\itemValue$ (which may not be integers),
and the size $\capacity$ of the knapsack.
The goal is to pick a subset of the items
that fits inside the knapsack
and maximizes the total value.

\newcommand{\knapsackAnswers}{

\vspace{0.5em}
\begin{enumerate}
\begin{minipage}{0.20\textwidth}
\item \ensuremath{\itemCount}
\end{minipage}
\hfill
\begin{minipage}{0.20\textwidth}
\item \ensuremath{\itemSize}
\end{minipage}
\hfill
\begin{minipage}{0.20\textwidth}
\item \ensuremath{\itemValue}
\end{minipage}
\hfill
\begin{minipage}{0.20\textwidth}
\item \ensuremath{\capacity}
\end{minipage}
\end{enumerate}

}

\begin{problemparts}

\problempart
\points{1}
Which input to the knapsack problem
corresponds to the input $\total$ in the stock purchasing problem?
\knapsackAnswers
\answerIId

\problempart
\points{1}
Which input to the knapsack problem
corresponds to the input $\stockCount$ in the stock purchasing problem?
\knapsackAnswers
\answerIIe

\problempart
\points{1}
Which input to the knapsack problem
corresponds to the input $\priceStart$ in the stock purchasing problem?
\knapsackAnswers
\answerIIf

\problempart
\points{1}
Which input to the knapsack problem
corresponds to the input $\priceEnd$ in the stock purchasing problem?
\knapsackAnswers
\answerIIg

\problempart
\points{6}
Unfortunately,
the algorithm for the knapsack problem
cannot be directly applied to the stock purchasing problem.
For each of the following potential reasons,
state whether it's a valid reason
not to use the knapsack algorithm.
(In other words, if the difference mentioned
were the only difference between the problems,
would you still be able to use the knapsack algorithm
to solve the stock purchasing problem?)

\begin{enumerate}
\item
In the stock purchasing problem,
there is a time delay between the selection
and the reward.

\item
All of the numbers in the stock purchasing problem are integers.
The $\itemValue$ array in the knapsack problem is not.

\item 
In the stock purchasing problem,
the money left over from your purchases
is kept as cash,
which contributes to your ultimate profit.
The knapsack problem has no equivalent concept.

\item
In the knapsack problem,
there are some variables representing
sizes of objects.
There are no such variables
in the stock purchasing problem.

\item
In the stock purchasing problem,
you can buy more than one share in each stock.

\item
In the stock purchasing problem,
you sell all the items at the end.
In the knapsack problem,
you don't do anything with the items.
 
\end{enumerate}
\answerIIh

\end{problemparts}

Despite these differences,
you decide that the knapsack algorithm
is a good starting point
for the problem you are trying to solve.
So you dig up some pseudocode for the knapsack problem,
relabel the variables to suit the stock purchasing problem,
and then start modifying things.
After a long night of work,
you end up with a couple of feasible solutions.
Unfortunately,
there is a bit of a hard-drive error the next morning,
and the files are all mixed up.
You have recovered six different functions,
from various states in your development process.
The first function is the following:

\begin{codebox}
\Procname{$\proc{Stock}
		(\total, \stockCount, \priceStart, \priceEnd)$}
\li $\purchase = \proc{Stock-Table}
		(\total, \stockCount, \priceStart, \priceEnd)$
\li \Return $\proc{Stock-Result}
		(\total, \stockCount, \priceStart, \priceEnd, \purchase)$
\end{codebox}

This is the function that you ran to get your results.
The $\proc{Stock-Table}$ function
generates the table of subproblem solutions.
The $\proc{Stock-Result}$ function
uses that to figure out which stocks to purchase,
and in what quantities.
Unfortunately,
you have two copies of the $\proc{Stock-Table}$ function
and three copies of the $\proc{Stock-Result}$ function.
You know that there's a way to take one of each function
to get the pseudocode for the original knapsack problem
(with the names changed).
You also know that there's a way to take one of each function
to get the pseudocode for the stock purchases problem.
You just don't know which functions do what.

\newcommand{\runtimeAnswers}{
\vspace{0.5em}
\begin{enumerate}
\begin{minipage}{0.49\textwidth}
\item \ensuremath{\Theta(\stockCount)}
\item \ensuremath{\Theta(\stockCount^2)}
\item \ensuremath{\Theta(\stockCount^3)}
\item \ensuremath{\Theta(\total)}
\item \ensuremath{\Theta(\total^2)}
\item \ensuremath{\Theta(\total^3)}
\end{minipage}
\hfill
\begin{minipage}{0.49\textwidth}
\item \ensuremath{\Theta(\stockCount + \total)}
\item \ensuremath{\Theta(\stockCount^2 + \total)}
\item \ensuremath{\Theta(\stockCount + \total^2)}
\item \ensuremath{\Theta(\stockCount \cdot \total)}
\item \ensuremath{\Theta(\stockCount^2 \cdot \total)}
\item \ensuremath{\Theta(\stockCount \cdot \total^2)}
\end{minipage}
\end{enumerate}
}

Analyze each of the other five procedures,
and select the correct running time.
Recall that $\total$ and $\stockCount$ are positive integers,
as are each of the values $\priceStart[\stock]$ and $\priceEnd[\stock]$.
To make the code simpler,
the arrays $\priceStart$, $\priceEnd$, and $\result$
are assumed to be indexed starting at $1$,
while the tables $\profit$ and $\purchase$
are assumed to be indexed starting at $(0, 0)$.
You may assume that entries in a table
can be accessed and modified in $\Theta(1)$ time.

\begin{figure}[p]
\begin{codebox}
\Procname{$\proc{Stock-Table-A}
		(\total, \stockCount, \priceStart, \priceEnd)$}
\li create a table $\profit$
\li create a table $\purchase$
\li \For $\cash \gets \const{0}$ to $\total$ \Do
\li		$\profit[\cash, \const{0}] \gets \cash$
\li		$\purchase[\cash, \const{0}] \gets \const{False}$
\li		\For $\stock \gets \const{1}$ to $\stockCount$ \Do
\li			$\profit[\cash, \stock] \gets
					\profit[\cash, \stock - \const{1}]$
\li			$\purchase[\cash, \stock] \gets \const{False}$
\li			\If $\priceStart[\stock] \le \cash$ \Then
\li				$\leftover \gets \cash - \priceStart[\stock]$
\li				$\current \gets \priceEnd[\stock] +
						\profit[\leftover, \stock]$
\li				\If $\profit[\cash, \stock] < \current$ \Then
\li					$\profit[\cash, \stock] = \current$
\li					$\purchase[\cash, \stock] = \const{True}$
	 			\End
			\End
		\End
	\End
\li \Return \purchase
\end{codebox}
\caption{The pseudocode for $\proc{Stock-Table-A}$.}
\label{fig:table-A}

\hfill

\begin{codebox}
\Procname{$\proc{Stock-Table-B}
		(\total, \stockCount, \priceStart, \priceEnd)$}
\li create a table $\profit$
\li create a table $\purchase$
\li \For $\cash \gets \const{0}$ to $\total$ \Do
\li		$\profit[\cash, \const{0}] \gets \const{0}$
\li		$\purchase[\cash, \const{0}] \gets \const{False}$
\li		\For $\stock \gets \const{1}$ to $\stockCount$ \Do
\li			$\profit[\cash, \stock] \gets
					\profit[\cash, \stock - \const{1}]$
\li			$\purchase[\cash, \stock] \gets \const{False}$
\li			\If $\priceStart[\stock] \le \cash$ \Then
\li				$\leftover \gets \cash - \priceStart[\stock]$
\li				$\current \gets \priceEnd[\stock] +
						\profit[\leftover, \stock - \const{1}]$
\li				\If $\profit[\cash, \stock] < \current$ \Then
\li					$\profit[\cash, \stock] = \current$
\li					$\purchase[\cash, \stock] = \const{True}$
	 			\End
			\End
		\End
	\End
\li \Return \purchase
\end{codebox}
\caption{The pseudocode for $\proc{Stock-Table-B}$.}
\label{fig:table-B}
\end{figure}

\begin{problemparts}
\problempart
\points{1}
What is the worst-case asymptotic running time of
$\proc{Stock-Table-A}$
(from Figure~\ref{fig:table-A})
in terms of $\stockCount$ and $\total$?
\runtimeAnswers

\answerIIi
\end{problemparts}

\begin{problemparts}
\problempart
\points{1}
What is the worst-case asymptotic running time of
$\proc{Stock-Table-B}$
(from Figure~\ref{fig:table-B})
in terms of $\stockCount$ and $\total$?
\runtimeAnswers

\answerIIj
\end{problemparts}

\begin{figure}[p]
\begin{codebox}
\Procname{$\proc{Stock-Result-A}
		(\total, \stockCount, \priceStart, \priceEnd, \purchase)$}
\li create a table $\result$
\li \For $\stock \gets \const{1}$ to $\stockCount$ \Do
\li		$\result[\stock] = \const{0}$
 	\End
\li
\li $\cash \gets \total$
\li $\stock \gets \stockCount$
\li \While $\stock > \const{0}$ \Do
\li		$\quantity \gets \purchase[\cash, \stock]$
\li		$\result[\stock] \gets \quantity$
\li		$\cash \gets \cash - \quantity \cdot \priceStart[\stock]$
\li		$\stock \gets \stock - \const{1}$
	\End
\li
\li	\Return $\result$
\end{codebox}

\caption{The pseudocode for $\proc{Stock-Result-A}$.}
\label{fig:result-A}

\hfill

\begin{codebox}
\Procname{$\proc{Stock-Result-B}
		(\total, \stockCount, \priceStart, \priceEnd, \purchase)$}
\li create a table $\result$
\li \For $\stock \gets \const{1}$ to $\stockCount$ \Do
\li		$\result[\stock] = \const{False}$
	\End
\li
\li $\cash \gets \total$
\li $\stock \gets \stockCount$
\li \While $\stock > \const{0}$ \Do
\li		\If $\purchase[\cash, \stock]$ \Then
\li			$\result[\stock] \gets \const{True}$
\li			$\cash \gets \cash - \priceStart[\stock]$
		\End
\li		$\stock \gets \stock - \const{1}$
	\End
\li
\li	\Return $\result$
\end{codebox}
\caption{The pseudocode for $\proc{Stock-Result-B}$.}
\label{fig:result-B}
\end{figure}

\begin{figure}

\begin{codebox}
\Procname{$\proc{Stock-Result-C}
		(\total, \stockCount, \priceStart, \priceEnd, \purchase)$}
\li create a table $\result$
\li \For $\stock \gets \const{1}$ to $\stockCount$ \Do
\li		$\result[\stock] = \const{0}$
	\End
\li
\li $\cash \gets \total$
\li $\stock \gets \stockCount$
\li \While $\stock > \const{0}$ \Do
\li		\If $\purchase[\cash, \stock]$ \Then
\li			$\result[\stock] \gets \result[\stock] + \const{1}$
\li			$\cash \gets \cash - \priceStart[\stock]$
\li		\Else
\li			$\stock \gets \stock - \const{1}$
		\End
	\End
\li
\li	\Return $\result$
\end{codebox}

\caption{The pseudocode for $\proc{Stock-Result-C}$.}
\label{fig:result-C}

\end{figure}

\begin{problemparts}
\problempart
\points{1}
What is the worst-case asymptotic running time of
$\proc{Stock-Result-A}$
(from Figure~\ref{fig:result-A})
in terms of $\stockCount$ and $\total$?
\runtimeAnswers

\answerIIk
\end{problemparts}

\begin{problemparts}
\problempart
\points{1}
What is the worst-case asymptotic running time of
$\proc{Stock-Result-B}$
(from Figure~\ref{fig:result-B})
in terms of $\stockCount$ and $\total$?
\runtimeAnswers

\answerIIl
\end{problemparts}

\begin{problemparts}
\problempart
\points{1}
What is the worst-case asymptotic running time of
$\proc{Stock-Result-C}$
(from Figure~\ref{fig:result-C})
in terms of $\stockCount$ and $\total$?
\runtimeAnswers

\answerIIm
\end{problemparts}

\begin{problemparts}

\problempart
\points{2}
The recurrence relation computed
by the $\proc{Stock-Table-A}$ function
is:
\begin{enumerate}
\item $\profit[\id{c}, \id{s}] = \displaystyle\max\{ \profit[\id{c}, \id{s}- 1], \profit[\id{c} - \priceStart[\id{s}], \id{s} - 1] \}$
\item $\profit[\id{c}, \id{s}] = \displaystyle\max\{ \profit[\id{c}, \id{s}- 1], \profit[\id{c} - \priceStart[\id{s}], \id{s} - 1] + \priceEnd[\id{s}]\}$
\item $\profit[\id{c}, \id{s}] = \displaystyle\max_q \{ \profit[\id{c} - q \cdot \priceStart[\id{s}], \id{s} - 1 ]+ q \cdot \priceEnd[\id{s}] \}$
\item $\profit[\id{c}, \id{s}] = \displaystyle\max\{ \profit[\id{c}, \id{s}- 1], \profit[\id{c} - \priceStart[\id{s}], \id{s} ] \}$
\item $\profit[\id{c}, \id{s}] = \displaystyle\max\{ \profit[\id{c}, \id{s}- 1], \profit[\id{c} - \priceStart[\id{s}], \id{s} ]+ \priceEnd[\id{s}] \}$
\item $\profit[\id{c}, \id{s}] = \displaystyle\max_q\{ \profit[\id{c} - q \cdot \priceStart[\id{s}], \id{s} ]+ q \cdot \priceEnd[\id{s}] \}$
\end{enumerate}

\answerIIn

\problempart
\points{2}
The recurrence relation computed
by the $\proc{Stock-Table-B}$ function
is:
\begin{enumerate}
\item $\profit[\id{c}, \id{s}] = \displaystyle\max\{ \profit[\id{c}, \id{s}- 1], \profit[\id{c} - \priceStart[\id{s}], \id{s} - 1] \}$
\item $\profit[\id{c}, \id{s}] = \displaystyle\max\{ \profit[\id{c}, \id{s}- 1], \profit[\id{c} - \priceStart[\id{s}], \id{s} - 1] + \priceEnd[\id{s}]\}$
\item $\profit[\id{c}, \id{s}] = \displaystyle\max_q \{ \profit[\id{c} - q \cdot \priceStart[\id{s}], \id{s} - 1 ]+ q \cdot \priceEnd[\id{s}] \}$
\item $\profit[\id{c}, \id{s}] = \displaystyle\max\{ \profit[\id{c}, \id{s}- 1], \profit[\id{c} - \priceStart[\id{s}], \id{s} ] \}$
\item $\profit[\id{c}, \id{s}] = \displaystyle\max\{ \profit[\id{c}, \id{s}- 1], \profit[\id{c} - \priceStart[\id{s}], \id{s} ]+ \priceEnd[\id{s}] \}$
\item $\profit[\id{c}, \id{s}] = \displaystyle\max_q\{ \profit[\id{c} - q \cdot \priceStart[\id{s}], \id{s} ]+ q \cdot \priceEnd[\id{s}] \}$
\end{enumerate}

\answerIIo

\end{problemparts}

With this information,
you should be able to figure out
whether
$\proc{Stock-Table-A}$
or
$\proc{Stock-Table-B}$
is useful for the knapsack problem,
and similarly for the stock purchasing problem.
From there,
you can figure out which of
$\proc{Stock-Result-A}$,
$\proc{Stock-Result-B}$, and
$\proc{Stock-Result-C}$
is best for piecing together
the optimal distribution of stocks and/or items.

\begin{problemparts}
\problempart
\points{3}
Which two methods, when combined,
let you compute the answer to the knapsack problem?
\begin{enumerate}
\item $\proc{Stock-Table-A}$ and $\proc{Stock-Result-A}$
\item $\proc{Stock-Table-A}$ and $\proc{Stock-Result-B}$
\item $\proc{Stock-Table-A}$ and $\proc{Stock-Result-C}$
\item $\proc{Stock-Table-B}$ and $\proc{Stock-Result-A}$
\item $\proc{Stock-Table-B}$ and $\proc{Stock-Result-B}$
\item $\proc{Stock-Table-B}$ and $\proc{Stock-Result-C}$
\end{enumerate}

\answerIIp

\problempart
\points{3}
Which two methods, when combined,
let you compute the answer to the stock purchases problem?
\begin{enumerate}
\item $\proc{Stock-Table-A}$ and $\proc{Stock-Result-A}$
\item $\proc{Stock-Table-A}$ and $\proc{Stock-Result-B}$
\item $\proc{Stock-Table-A}$ and $\proc{Stock-Result-C}$
\item $\proc{Stock-Table-B}$ and $\proc{Stock-Result-A}$
\item $\proc{Stock-Table-B}$ and $\proc{Stock-Result-B}$
\item $\proc{Stock-Table-B}$ and $\proc{Stock-Result-C}$
\end{enumerate}

\answerIIq

\end{problemparts}

With all that sorted out,
you submit the code to your supervisor
and pat yourself on the back
for a job well done.
Unfortunately, your supervisor comes back
a few days later
with a complaint
from the higher-ups.
They've been playing with your program,
and were very upset to discover that when they ask what to do
with \$1,000,000,000 in the year 1991,
it tells them to buy tens of millions of shares
in Dale, Inc.
According to them,
there weren't that many shares of Dale
available to purchase.
They want a new feature:
the ability to pass in limits
on the number of stocks purchaseable.

You choose to begin,
as always,
with a small example:
\begin{center}
\begin{tabular}{|l|c|c|c|}
\hline
\bf Company & \bf Price in 1991 & \bf Price in 2011 & \bf Limit \\ \hline
Dale, Inc. & \$12 & \$39 & 3 \\
JCN Corp. & \$10 & \$13 & $\infty$ \\
Macroware, Inc. & \$18 & \$47 & 2 \\
Pear, Inc. & \$15 & \$45 & 1 \\ \hline
\end{tabular}
\end{center}

\begin{problemparts}

\problempart
\points{5}
If you had \$30 available to purchase stocks in 1991,
how much of each stock should you have bought,
given the limits imposed above?

\solution{

\begin{center}
\begin{tabular}{|c|c|}
\hline
\bf Company & \bf Number of Shares \\
\hline
Dale, Inc. & \answerIIrDale \\
JCN Corp. & \answerIIrJCN \\
Macroware, Inc. & \answerIIrMacroware \\
Pear, Inc. & \answerIIrPear \\
\hline
\end{tabular}
\end{center}
}

\problempart
\points{5}
If you had \$120 available to purchase stocks in 1991,
how much of each stock should you have bought,
given the limits imposed above?

\solution{

\begin{center}
\begin{tabular}{|c|c|}
\hline
\bf Company & \bf Number of Shares \\
\hline
Dale, Inc. & \answerIIsDale \\
JCN Corp. & \answerIIsJCN \\
Macroware, Inc. & \answerIIsMacroware \\
Pear, Inc. & \answerIIsPear \\
\hline
\end{tabular}
\end{center}
}

\problempart
\points{20}
Give pseudocode
for an algorithm $\proc{StockLimited}$
that computes the maximum profit achievable
given
a starting amount $\total$,
a number $\stockCount$ of companies with stock available,
an array of initial prices $\priceStart$,
an array of final prices $\priceEnd$,
and an array of quantities $\limit$.
The value stored at $\limit[\stock]$
will be equal to $\infty$
in cases where there is no known limit
on the number of stocks.
The algorithm need only output
the resulting quantity of money,
not the purchases necessary to get that quantity.

Remember to analyze the runtime of your pseudocode,
and provide a brief justification for its correctness.
It is sufficient to give
the recurrence relation that your algorithm implements,
and talk about why the recurrence relation
solves the problem at hand.

\answerIIt
\end{problemparts}

\end{problems}

\end{document}
